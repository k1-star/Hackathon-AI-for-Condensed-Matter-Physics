\documentclass[12pt,a4paper]{article}
\usepackage[utf8]{inputenc}
\usepackage{ctex}
\usepackage{hyperref}
\usepackage{geometry}
\geometry{left=2.5cm,right=2.5cm,top=2.5cm,bottom=2.5cm}

\title{基于机器学习的材料发现项目可行性方案}
\author{gogogo}
\date{\today}

\begin{document}
\maketitle


\section{项目框架}
\subsection{问题切入}
\begin{itemize}
    \item 选择一个我们研究的\textbf{核心性质},例如:带隙、形成能、超导转变温度($T_c$)。
    \item 定义\textbf{模型评判标准},例如:稳定半导体,要求带隙 $>$ 2 eV 且形成能接近凸包。
\end{itemize}

\subsection{数据收集}
从公开的第一性原理数据库获取材料结构与性质数据:
\begin{itemize}
    \item \href{https://materialsproject.org}{Materials Project (MP)},支持API调用和pymatgen工具。
    \item \href{http://oqmd.org}{OQMD (Open Quantum Materials Database)}。这玩意是最权威的之一但是我打不开链接
    \item \href{http://www.aflowlib.org}{AFLOW (Automatic Flow for Materials Discovery)}。
    \item \href{https://jarvis.nist.gov}{JARVIS-DFT (NIST)}。
    \item \href{https://nomad-lab.eu}{NOMAD (Novel Materials Discovery Laboratory)}。
\end{itemize}

\subsection{数据预处理}
\begin{itemize}
    \item 清理重复数据和缺失值。
    \item 材料特征化方法:
    \begin{itemize}
        \item 基于成分:Magpie特征、mat2vec嵌入(使用\texttt{matminer}工具包)。
        \item 基于晶体结构:图神经网络(GNN)构建原子-键的周期图。
    \end{itemize}
\end{itemize}

\subsection{基线模型}
\begin{itemize}
    \item 使用简单监督学习模型进行基线实验:
    \begin{itemize}
        \item 随机森林(Random Forest)。
        \item XGBoost。
    \end{itemize}
    \item 在验证集上评估性能(MAE、$R^2$),作为后续模型的比较基准。
\end{itemize}

\subsection{图神经网络模型}
\begin{itemize}
    \item \textbf{CGCNN}:晶体图卷积神经网络。
    \item \textbf{MEGNet}:引入状态变量和迁移学习。
    \item \textbf{ALIGNN}:利用线图结构引入键角信息。
    \item \textbf{M3GNet}:引入三体相互作用,可作为通用势能面。
\end{itemize}

\subsection{主动学习与筛选}
\begin{itemize}
    \item 使用训练好的代理模型预测大规模候选材料的性质。
    \item 引入\textbf{不确定性估计}(如集成模型、dropout)来识别有前景但模型不确定的样本。
    \item 结合\textbf{贝叶斯优化}或主动学习迭代,与DFT验证形成闭环。
\end{itemize}

\subsection{最终交付}
\begin{itemize}
    \item 一个完整的模型(基线 + GNN)。
    \item 一份候选材料清单,满足预设目标。
    \item 可视化结果:
    \begin{itemize}
        \item 特征重要性排序图。
        \item 预测值 vs 真实值散点图。
        \item 候选材料排名表格。
    \end{itemize}
\end{itemize}

\section{扩展功能}
在基本框架之上,可以进一步加入:
\begin{itemize}
    \item 结合文本优化,利用文献进行相关延伸或补充。
    \item 迁移学习(在大数据库上预训练,在小数据集上微调)。这一点我们就可以根据交大的强势方向进行特化,比如针对电池材料,二维超导材料等特定领域进行微调。
    \item 生成模型(如VAE、GAN)实现材料逆向设计。
\end{itemize}


\end{document}

